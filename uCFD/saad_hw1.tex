\documentclass{article}
\usepackage{graphicx}
\usepackage{amsmath}
\graphicspath{ {/Users/mattwilliams/Desktop/} }

\begin{document}
\begin{titlepage}
	\begin{center}
	\line(1,0){300} \\ % Slope of line and length. Double escape character adds a break after the line
	[0.25in]               % Add spacing to the line above the title
	\huge{\bfseries Homework 1, Tony Saad's uCFD Course} \\
	[2mm]
	\line(1,0){200} \\
	[1.5cm]
	\textsc{\LARGE From University of Utah CHEN 6355} \\
	[1cm]
	\includegraphics[scale=0.75]{advection} \\
	[4cm]
	\end{center}
	\begin{flushright}
	\textsc{\large Matt Williams} \\
	matt.williams@alum.mit.edu
	\end{flushright}
\end{titlepage}
\section{Question 1: The Energy Equation}\label{sec:Q4}
\textit{What is a thermodynamic relation between temperature and enthalpy?} \\ \\
The purpose of this question is to develop an equation for temperature from the Navier-Stokes energy equation: \\
$$\frac{\partial \rho h}{\partial t} =\frac{D\rho }{Dt} -\nabla \cdot \textbf{u}\rho h\  -\tau_{ij} \frac{\partial u_{i}}{\partial x_{j}} -\nabla \cdot \textbf{q}$$
Using the definition of the specific heat capacity at a constant temperature:
$$\frac{dh}{dT} =c_{p}$$
we can create a discretized expression for temperature in terms of enthalpy.
$$\int^{h_{f}}_{h_{i}} dh=\int^{Tf}_{T_{i}} c_{p}dT$$
$$T_{f}=T_{i}+\frac{1}{c_{p}} (h_{f}-h_{i})$$
\textit{Write out the $\tau_{ij} \frac{\partial u_{i}}{\partial x_{j}}$ term. This term is the viscous dissipation. Note that repeated indices indicate a summation over all indices.}\\ \\
Expanding the term as indicated above yields:
$$\tau_{11} \frac{\partial u_{1}}{\partial x_{1}} +\tau_{12} \frac{\partial u_{1}}{\partial x_{2}} +\tau_{21} \frac{\partial u_{2}}{\partial x_{1}} +\tau_{22} \frac{\partial u_{2}}{\partial x_{2}} $$
Converting this expression to drop the indices for the velocity and spatial dimensions:
$$\tau_{11} \frac{\partial u}{\partial x} +\tau_{12} \frac{\partial u}{\partial y} +\tau_{21} \frac{\partial v}{\partial x} +\tau_{22} \frac{\partial v}{\partial y} $$
Now, to expand on the definition of the stress tensor, that is:
$$\tau_{ij} =-\mu \left( \frac{\partial u_{i}}{\partial x_{j}} +\frac{\partial u_{j}}{\partial x_{i}} \right)  +\frac{2}{3} \mu \delta_{ij} \nabla \cdot \textbf{u}$$
The diagonal terms contain the extra contribution from the dirac delta function, and are expanded as:
$$\tau_{11} =-\frac{4}{3} \mu \frac{\partial u}{\partial x} +\frac{2}{3} \mu \frac{\partial v}{\partial y} $$
$$\tau_{22} =-\frac{4}{3} \mu \frac{\partial v}{\partial y} +\frac{2}{3} \mu \frac{\partial u}{\partial x} $$
The off-diagonal terms contain only the contribution from the first term:
$$\tau_{12} =-\mu \left( \frac{\partial u}{\partial y} +\frac{\partial v}{\partial x} \right) $$ 
$$\tau_{21} =-\mu \left( \frac{\partial v}{\partial x} +\frac{\partial u}{\partial y} \right)  $$
which are the same due to the symmetry of the stress tensor. Completing the multiplications for each term and simplifying results in:
$$-\frac{4}{3} \mu \left[ \left( \frac{\partial u}{\partial x} \right)^{2}  -\frac{\partial u}{\partial x} \frac{\partial v}{\partial y} +\left( \frac{\partial v}{\partial y} \right)^{2}  \right]  -\mu \left( \frac{\partial u}{\partial y} +\frac{\partial v}{\partial x} \right)^{2}  $$ \\
\textit{Assume Fourier's Law for heat transfer, i.e. $\textbf{q} = -k\nabla T$}\\
Inserting Fourier's law, and the definition of the specific heat capacity:
$$\rho c_{p}\frac{\partial T}{\partial t} =\frac{\partial p}{\partial t} +p\nabla \cdot \textbf{u}-\rho c_{p}T\nabla \cdot \textbf{u}+\frac{4}{3} \mu \left[ \left( \frac{\partial u}{\partial x} \right)^{2}  -\frac{\partial u}{\partial x} \frac{\partial v}{\partial y} +\left( \frac{\partial v}{\partial y} \right)^{2}  \right]  +\mu \left( \frac{\partial u}{\partial y} +\frac{\partial v}{\partial x} \right)^{2}  +k\nabla^{2} T$$
A further simplification can be made if the fluid is assumed to be an ideal gas. In that case, the substitution $p = \rho RT$ can be made, getting another expression for the time derivative of the temperature.

\section{Question 2: Vorticity-streamfunction Navier-Stokes}\label{sec:Q2}
\textit{If a function $\psi$ exists such that $u\  =\  \frac{\partial \psi }{\partial y} ,\  v=-\frac{\partial \psi }{\partial y} $, show that continuity is satisfied by $\psi$.} \\
Substituting the definition of the streamfunction for each velocity into the continuity equation:
$$\frac{\partial }{\partial x} \frac{\partial \psi }{\partial y} -\frac{\partial }{\partial y} \frac{\partial \psi }{\partial x} =0$$
$$\frac{\partial^{2} \psi }{\partial x\partial y} -\frac{\partial^{2} \psi }{\partial x\partial y} =0$$
\textit{Derive an equation from the NS momentum equations that are a function only of vorticity and the streamfunction.} \\
By differentiating (5) with respect to y and (6) with respect to x, we obtain:
$$\frac{\partial^{2} u}{\partial y\partial t} +\left( \frac{\partial u}{\partial x} \frac{\partial u}{\partial y} +u\frac{\partial^{2} u}{\partial x\partial y} \right)  +\left( \frac{\partial v}{\partial y} \frac{\partial u}{\partial y} +v\frac{\partial^{2} v}{\partial y^{2}} \right)  =-\frac{1}{\rho } \frac{\partial^{2} p}{\partial x\partial y} +\frac{\mu }{\rho } \left( \frac{\partial^{3{}} u}{\partial x^{2}\partial y} +\frac{\partial^{3} u}{\partial y^{3}} \right)   $$
$$\frac{\partial^{2} v}{\partial x\partial t} +\left( \frac{\partial u}{\partial x} \frac{\partial v}{\partial x} +u\frac{\partial^{2} v}{\partial x^{2}} \right)  +\left( \frac{\partial v}{\partial x} \frac{\partial v}{\partial y} +v\frac{\partial^{2} v}{\partial x\partial y} \right)  =-\frac{1}{\rho } \frac{\partial^{2} p}{\partial x\partial y} +\frac{\mu }{\rho } \left( \frac{\partial^{3{}} v}{\partial x^{3}} +\frac{\partial^{3} v}{\partial x\partial y^{2}} \right)   $$
Subtracting the latter equation from the former:
\begin{equation*}
\begin{aligned}
\frac{\partial }{\partial t} \left( \frac{\partial v}{\partial x} -\frac{\partial u}{\partial y} \right)  +u\frac{\partial }{\partial x} \left( \frac{\partial v}{\partial x} -\frac{\partial u}{\partial y} \right)  +v\frac{\partial }{\partial y} \left( \frac{\partial v}{\partial x} -\frac{\partial u}{\partial y} \right)  +\left( \frac{\partial u}{\partial x} +\frac{\partial v}{\partial y} \right)  \left( \frac{\partial v}{\partial x} -\frac{\partial u}{\partial y} \right)  =\\ \frac{\mu }{\rho } \left( \frac{\partial^{2} }{\partial x^{2}} \left[ \frac{\partial v}{\partial x} -\frac{\partial u}{\partial y} \right]  +\frac{\partial^{2} }{\partial y^{2}} \left[ \frac{\partial v}{\partial x} -\frac{\partial u}{\partial y} \right]  \right) 
\end{aligned}
\end{equation*}
Using the definition of vorticity, $\omega =\nabla \times \textbf{u}=\left( \frac{\partial v}{\partial x} -\frac{\partial u}{\partial y} \right)  \textbf{j}$, the above equation becomes:
$$\frac{\partial \omega }{\partial t} +\frac{\partial \psi }{\partial y} \frac{\partial \omega }{\partial x} -\frac{\partial \psi }{\partial x} \frac{\partial \omega }{\partial y} =\frac{\mu }{\rho } \left( \frac{\partial^{2} \omega }{\partial x^{2}} +\frac{\partial^{2} \omega }{\partial y^{2}} \right)  $$
where we have taken advantage of the fact that the fourth term on the left-hand side contains the divergence of the velocity, which is zero for the case of a constant density. Because the vorticity is equal to a Poisson equation for the streamfunction:
$$\omega =\frac{\partial v}{\partial x} -\frac{\partial u}{\partial y} =-\frac{\partial^{2} \psi }{\partial x^{2}} -\frac{\partial^{2} \psi }{\partial y^{2}} $$
you would be able to find an initial streamfunction given an initial condition for the vorticity. An iteration strategy could be the following:
\begin{itemize}
	\item Find streamfunction from initial condition for vorticity using the Poisson equation
	\item Find the next timestep for vorticity using the NS equation
	\item Update the streamfunction using the new condition for vorticity
	\item Repeat
\end{itemize}
\end{document}